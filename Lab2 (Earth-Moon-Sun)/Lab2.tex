\documentclass[12pt]{article}% uses letterpaper by default

%---------- Uncomment one of them ------------------------------
\usepackage[includeheadfoot, top=1in, bottom=1in, hmargin=1in]{geometry}

% \usepackage[a5paper, landscape, twocolumn, twoside,
%    left=2cm, hmarginratio=2:1, includemp, marginparwidth=43pt, 
%    bottom=1cm, foot=.7cm, includefoot, textheight=11cm, heightrounded, columnsep=1cm, dvips,  verbose]{geometry}
%---------------------------------------------------------------
\usepackage{fancyhdr}
\renewcommand{\footrulewidth}{0.4pt}% default is 0pt
\usepackage{verbatim}
\usepackage{url}
\usepackage{cancel}
\pagestyle{fancy}
\usepackage{graphicx}
\usepackage{setspace}
\usepackage{hyperref}
\usepackage{csquotes}
\singlespacing
\usepackage{varwidth}

\usepackage{indentfirst} % indent the first paragraph of a section
\usepackage{amsmath}
\usepackage{epigraph}

\lhead{Astronomy Lab I}
\rhead{Fall 2021}
\lfoot{Alina Sabyr}

\cfoot{\thepage}

\begin{document}

\begin{center}
    \LARGE Lab 2: The Earth-Moon-Sun System
\end{center}

Today you will explore Earth's rotation and its revolution around the Sun; you will hopefully leave lab today with a scientific and visual understanding of what causes the seasons. You'll also learn about the Moon, which is the Earth's only natural satellite and the brightest object in the night sky (making is a frequent annoyance to astronomers).  Weather permitting, we'll get to do some observing.

\section{Introduction}

We will start with a quick warm-up exercise (you can discuss the following questions with your neighbors or do on your own). Write down the answers in your notebook.
\begin{enumerate}
\item How long does it take the Earth to rotate once on its own axis?
\item How long does it take the Earth to revolve once around the Sun?
\item The Sun rises in the East and sets in the West. Why?
\item What causes the seasons?
\item What causes the Moon's phases?
\end{enumerate}


\section{Earth's Rotation}

In this section, you will model the Earth's rotation.  The lamp is the sun, and, for now, your head will represent Earth.  Here are some preliminary questions. Write your answers in your notebook.
\begin{itemize}
	\item Where is the North pole in this model? The South pole?
	\item Where is the equator?
	\item How do you experience 'daytime' and 'nighttime' in this model?
	\item How can you control the length of the day in your model?
\end{itemize}

Now, pretend there is a microbe-sized person standing on the tip of your nose.  Their feet are flat on your nose, and they are facing the floor.

\begin{enumerate}
    \item What is directly over this person's head?
	\item Where is their horizon? 
	\item Where should the Sun rise and set for them? (Which way is East and which way is West?) State which direction should the Earth turn in for this to be true (clockwise or counterclockwise if looking from overhead).
\end{enumerate}


Determine what time it is in your city (i.e., your nose) when you stand in the following positions:
\begin{enumerate}
	\item Facing the Sun
	\item Facing directly away from the Sun 
	\item With your right shoulder towards the Sun
    \item In your notebook, sketch a diagram showing your ``Face-Earth'' and the Sun from above, and indicate clearly on the diagram which direction you would need to be facing in order for it to be the following times in your ``Nose-City'': 6 am, 9 am, Noon, 3 pm, 6 p.m., 9 p.m., midnight, 3 a.m.  
\end{enumerate}

\section{The Moon's Phases}
Now we'll return to our personal model from Section 1 and discuss the phases of the Moon. Everyone will get a styrofoam ball with a stick -- this is your moon, and your head will again be Earth. 

Hold the ``Moon'' and rotate around the ``Earth'' counterclockwise until you complete a full circle.  While doing this, the ``Earth'' will rotate while always facing the ``Moon'' to see what the ``Moon'' looks like at any given time from the ``Earth.'' Note which part of the Moon is bright and which is dark at any given time. 

\begin{enumerate}
    \item In the moon-phase diagram (I will pass this out), draw the shape of the Moon in each position. Make sure that you clearly mark which is the bright and which is the dark part of the Moon in your drawings. 
    \item Can you recognize the phases of the Moon? Which configuration gives a Full Moon? Which gives a New Moon (when you can only see the dark side of the Moon)? Indicate them in your diagram.
    \item Think about the earth's rotation.  Write what time the Moon rises and sets at when it's at the positions labelled 1, 3, 5 and 7.
\end{enumerate}
Write your name and turn in this diagram with your report.

\section{Earth's Revolution}
Okay, time to use the globes: they will represent...well, guess. We'll use the globes to see how the amount of sunlight hitting different regions of the Earth changes as Earth goes around the sun. Keep the North pole of the globe pointed at the ``North Star'' (the Northern wall/ceiling of the library), and note that Earth's axis of rotation is slightly tilted with respect to its orbital plane around the Sun.
\begin{enumerate}
\item The globe has New York marked on it. First, determine which way you need to spin the globe in order for sunrise and sunset to appear in the right directions. Make a diagram in your notebook showing your setup from above, as well as the direction of Earth's rotation around its own axis. Make sure to leave enough room in your diagram for the Earth to revolve all the way around the Sun! Also, indicate in your diagram which way your model is set up with respect to the classroom.
\item Find the points in Earth's orbit around the Sun where the Northern Hemisphere points toward and away from the Sun. Label those points clearly in your diagram, adding a small sketch that illustrates how the Earth is tilted with respect to the Sun in each location.
\item You'll notice another city marked on the globe, directly south of New York. Use these two cities to compare the relative times of sunrise/sunset and the relative lengths of the days in each hemisphere at both points in Earth's orbit.
\item Move the globe around the Sun until you find the points in Earth's orbit where the following four events happen in New York City. Note that the Earth orbits the Sun counterclockwise as viewed from the Northern hemisphere.
    \begin{itemize}
    \item Winter Solstice (shortest day of the year)
    \item Summer Solstice (longest day of the year)
    \item Vernal (Spring) Equinox (day and night are equal length)
    \item Autumnal (Fall) Equinox (day and night are equal length)
    \end{itemize}
\item In your notebook, sketch the Sun and show Earth's position at each of the four locations above (you should have already added two of these positions to your diagram in one of the previous questions -- if so, just add the other two). Add a small sketch that illustrates how the Earth is tilted with respect to the Sun in each location, and be sure to label Earth's axis of rotation and both hemispheres.
\item On the equinoxes, which city experiences a longer day?
\item What causes the seasons on Earth?
\item Over a full year, does one of the cities receive more sunlight than another?
\item Why is the average temperature at the equator higher than it is at the North Pole?
\end{enumerate}

\section{Eclipses}
Now that you are familiar with the motion of the Moon, think about eclipses.
\begin{enumerate}
    \item What configuration of the Sun-Earth-Moon system could result in a lunar eclipse (in which Earth’s shadow falls on the Moon)? A solar eclipse (in which the Moon’s shadow falls on Earth)? Make a sketch in your notebook in each case.
    \item What is the phase of the Moon during a solar eclipse? During a lunar eclipse?
    \item Why do you think eclipses don't happen every month? 
\end{enumerate}

\section{Observing}
Now we will go observing! We will split into two groups. Observing TA (Daniel) will help you with this part of the lab. 

In your notebook answer the following. 
\begin{enumerate}
    \item What is the phase of the moon tonight? Does that make it a good night for us to observe? (think about the connection between the moon phase and the time it rises and sets).
    \item Describe what you saw. Besides magnification, did the Moon look different through the telescope?
    \item As you observe the Moon, you will need to keep moving the telescope. Why is that?
\end{enumerate}

\section{Conclusions}
\begin{enumerate}
    \item What did you like or dislike in this lab?
    \item Is anything still confusing?
    \item What is a question or comment you have about today’s lab?
\end{enumerate}

\end{document}