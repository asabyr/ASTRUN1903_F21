\documentclass[10pt]{article} 

\usepackage[includeheadfoot, top=1in, bottom=1in, hmargin=1in]{geometry}

%\usepackage{amsmath, amsfonts, amssymb,epsfig,graphicx}

\usepackage{fancyhdr}
\usepackage{url}
\pagestyle{fancy}
\usepackage{setspace}


%\doublespacing
\singlespacing
%\onehalfspacing

\lhead{Astronomy Lab I}
% \chead{Lab 1}
\rhead{Fall 2021}
% \lfoot{Yavetz}
\cfoot{\thepage}
\rfoot{}

\begin{document}

\begin{center}
{\huge Astronomy Lab I: Grading Rubric}\\
\medskip
% {\Large September 12th, 2018}
\end{center}

\bigskip

\section{Lab Write-Up Guidelines}

The primary goal of this class is to teach you how science is actually done. This means, in part, keeping a record of everything. Please write down everything you do and in the order you do it. State assumptions, show work for calculations, and so on. You will do most labs with a partner, but make sure to keep your own records. I should be able to reproduce your answers with just the information in your notebook. Below are some formatting and specific content requests.

\begin{itemize}
\item Begin each lab writeup on a new page (or document) and have your name, your partner's name, the lab title, and the date at the top.
\item Always include units on numbers with units, and always label plot axes.
\item Put a box (or highlight) around numerical answers (and make sure to show your work!)
\end{itemize}

\section{Grading Rubric}

Each lab write-up will be assigned a grade out of 10. The points will be assigned based on three dimensions:

\begin{enumerate}
\item \textbf{Clarity of writing:} How you explain and justify your answers.
\begin{itemize}
	\item 1 point: Little or no justification of answers, hard to follow explanations or logic.
	\item 2-3 points: Explanations are reasonable and the logic holds up, but additional clarification may be in order, scientific terms are used approriately most of the time, pre- and post-lab reflections are completed for every lab.
	\item 4 points: Explanations are clear and every answer is fully justified, scientific terms are used correctly, and pre- and post-lab reflections are thoughtful and raise additional questions about the subject matter.
\end{itemize}
\item \textbf{Use of graphs, diagrams, and equations:} A picture (or a mathematical equation) can be worth 1000 words!
\begin{itemize}
	\item 1 point: Graphs, diagrams, and equations are rarely used where appropriate, not well explained, or lack axis labels or legends.
	\item 2 points: A valid attempt is made to use graphs, diagrams, and equations appropriately, though they are not always used in the correct context or they are not fully explained.
	\item 3 points: Graphs, diagrams, and equations are used appropriately, are well explained, and relate directly to the rest of the lab assignment.
\end{itemize}
\item \textbf{Correctness of answers:} You don't always have to be right!
\begin{itemize}
	\item 1 point: The answers are incorrect and off by many orders of magnitude, in a way that should have made it clear that the answer is incorrect.
	\item 2 points: The answers are incorrect, but could be reasonable given the question. Alternatively, the answers are incorrect and off by many orders of magnitude, but are identified as such and a hypothesis is made as to why they are off by so much.
	\item 3 points: The answers are correct to within an order of magnitude (or to the degree of accuracy specified by the specific question).
\end{itemize}
\end{enumerate}


\end{document}