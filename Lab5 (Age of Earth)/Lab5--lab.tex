\documentclass[12pt]{article}% uses letterpaper by default

%---------- Uncomment one of them ------------------------------
\usepackage[includeheadfoot, top=1in, bottom=1in, hmargin=1in]{geometry}

% \usepackage[a5paper, landscape, twocolumn, twoside,
%    left=2cm, hmarginratio=2:1, includemp, marginparwidth=43pt, 
%    bottom=1cm, foot=.7cm, includefoot, textheight=11cm, heightrounded, columnsep=1cm, dvips,  verbose]{geometry}
%---------------------------------------------------------------
\usepackage{fancyhdr}
\renewcommand{\footrulewidth}{0.4pt}% default is 0pt
\usepackage{verbatim}
\usepackage{url}
\usepackage{cancel}
\pagestyle{fancy}
\usepackage{graphicx}
\usepackage{setspace}
\usepackage{hyperref}
\usepackage{csquotes}
\usepackage{enumitem}
\singlespacing
\usepackage{varwidth}

\usepackage{indentfirst} % indent the first paragraph of a section
\usepackage{amsmath}
\usepackage{epigraph}

\lhead{Astronomy Lab I}
\rhead{Fall 2021}
\cfoot{\thepage}

\usepackage{xcolor}
\newcommand{\note}[1]{\textcolor{red}{(#1)}}
\begin{document}

\begin{center}
    \LARGE Lab 9: The Age of the Earth
\end{center}

The age of the Earth has been the subject of much debate since the at least the 16th century. 
In the 1580s, Bernard Palissy suggested that the Earth must be quite old because erosional forces (i.e. rivers carving canyons) work over time periods longer than the few thousand years covered by historical records. 
(He was burned at the stake in 1589.)
James Hutton in Scotland around 1770 followed a similar train of thought; erosion was unable to alter Hadrian's Wall over 1500 years after the Romans built it so probably the mountains are older than the 6000 years the Catholic Church claimed given the wear evident on them. 
More recently, tree ring records and ice cores can go back tens of thousands of years; we need to get much further.  
Geological evidence shows than many areas were once covered in oceans (some several times), and creatures that are now extinct are contained in great striations of sedimentation.
But how long does it take to put down each layer of this clearly ancient rock? 
There are many potential ways to attack this problem, including thermodynamic arguments, plate tectonics, cosmic ray tracks in crystals, and the mutation rate of DNA. In this lab, you will split up into groups and attempt to do it yourselves!


\section{Calculate the age of the Earth}

\begin{enumerate}

    \item \textbf{Premade Problem}: (6-6:45pm) I will give each group a pre-written method for calculating the age of the earth, which you will work on in conjunction with your own research. 
    \begin{itemize}
        \item In your writeup, address how accurate do you think this method is.
        \item Also, consider applying this technique to a different planet, both in our solar system and around another star. Would it still work? Why or why not? What else would you need to know?
    \end{itemize} 
    
    \item \textbf{Develop Ideas, Research}: (6:45-7:45pm) Following up on the first part of the class, now come up with your own method for estimating the age of the Earth. Before you look anything up, talk to each other and come up with an idea for how you would answer the question ``how old is the Earth?''.   Think about what information you need and write down exactly how you would use it to get an answer. When you are satisfied with your proposed method, run it by me so I make sure it sounds reasonable and feasible. Look up the information you need and refine your technique(s) as necessary. \textbf{Any numerical answers you ultimately find are not too important: I am most interested in your logical and scientific reasoning.} You may be starting over many times if you realize that one of your proposed methods will not or does not work: that's okay! Part of scientific thinking is refining your ideas as you gather new evidence. Make sure to record the following:
    \begin{itemize}
        \item Your method, in detail.
        \item Any information you look up.
        \item The limitations of your method(s). What limitations could be overcome with better equipment or data?
        
        \item How accurate you believe your method are compared to what I gave you.
        \item Would it work for a different planet, both in our solar system and around another star? Why or why not?
    \end{itemize}  

    \item \textbf{Presentations}: (7:45-8:45pm) Each group will present their results to the others. 
    \begin{itemize}
        \item Present in a logical order how your method(s) and the one I gave you work, and discuss any thoughts you have on accuracy, complexity, etc.
        \item Presentations should be approximately 10 minutes. 
        \item Try to have everyone in the group present some aspect of your procedure/results.
    \end{itemize} 
\end{enumerate}

\section{Conclusion}
(Do this part by yourself)\\
\begin{itemize}
    \item Write a short summary of the other two group's methods after they present.  Prove to me that you were paying attention!
    \item What was your favorite and least favorite of this lab?
    \item Is anything still unclear?
    \item Ask a question. 
\end{itemize}

\end{document}

