\documentclass[12pt]{article} 
 
\usepackage[includeheadfoot, top=0.75in, bottom=0.5in, hmargin=0.75in]{geometry}
 
%\usepackage{amsmath, amsfonts, amssymb,epsfig,graphicx}
 
\usepackage{fancyhdr}
\usepackage{url}
\pagestyle{fancy}
\usepackage{setspace}
\usepackage{indentfirst}

 
%\doublespacing
\singlespacing
%\onehalfspacing
 
\lhead{ASTR UN1903}
\rhead{Fall 2021}
\cfoot{\thepage}
\rfoot{}
 
\begin{document}
 
\begin{center}
{\huge ASTRONOMY LAB I}\\

\end{center}
 
\bigskip
 
 
\noindent \textbf{\normalsize Instructor:} {\normalsize Alina Sabyr} \\ 
\textbf{\normalsize Email:} {\normalsize \url{as6131@columbia.edu}}\\
 
 
\noindent \textbf{\normalsize Time/Location:} {\normalsize Mondays 6-9 PM in Pupin 1402} \\
\textbf{\normalsize Office hours:} {\normalsize by appointment}
 
 
 
\section*{Class overview}
 
Welcome to the astronomy lab! In this lab you will learn about scientific method and further develop critical thinking. 
 
There will be a total of 11 lab sessions (9 Labs + final presentations). During the 10th lab session you will work on your presentations and during the final lab session you will present to the class ($\approx$ 10 min long presentations followed by 5 min long discussion). More information about the format and topics of these presentations will be posted later in the semester. There will be no work assigned outside of these lab sessions. We will also be observing for 2-3 lab sessions this semester (weather permitting). 

Please don't hesitate to ask lots of questions and hope you learn a lot and enjoy lab this semester!
 
\section*{Lab Materials}
 
Please bring the following to each lab session:
 
\begin{itemize}
\item \textbf{Lab Notebook/Paper + Pen/Pencil or Tablet/Laptop}: I will collect your completed lab write-up at the end of each session and return it to you at the beginning of the next lab.

\item \textbf{Scientific Calculator or Laptop}:  A calculator capable of doing trigonometric functions, logs, exponents, roots, etc. This does not need to be a graphing calculator. Alternatively you can use your laptop too.
 
\item \textbf{Warm Clothes}: We will be observing on the roof of Pupin for 2-3 labs, which is usually cold and windy, especially later in the semester. Since observing is weather-dependent and cannot always be predicted, please bring layers to each lab session (except the first one)! A rule of thumb is to dress for 20°F below the actual temperature. 
 
\end{itemize}
 
\section*{Grading}

\subsubsection*{$90\%$ Lab Write-ups (+ final presentation):}
\noindent During each lab, I'll make clear what I'm looking for you to record in your lab notebooks (see below). Grading rubric is posted on courseworks. If you are working with a partner, each of you should keep your own records. 
\subsubsection*{Lab write-up guidelines:}
 
\begin{itemize}
\item[--] Each lab write-up should begin on a new page and have your name, your partner's name, lab
title, and the date at the top.
\item[--] State specifically and in detail what your assumptions, methods, calculations, observations and
conclusions are.
\item[--] Put a box around your final answer. However, note that the exact value is lesser important than the methods you used to get there (above).
\item[--] Always include units! Writing them in at each stage of calculations will help you keep track. Your
units should be appropriate, e.g. don't measure the Sun in centimeters unless specifically asked
to do so.
\item[--] Plots should have both axes labelled with units, and a legend or other indication of what each
symbol/line represents
\item[--] Ensure your handwriting is legible. 
\item[--] For the final lab session, your 10-minute presentation will take the place of a write-up.
\end{itemize}
 
\subsubsection*{$10\%$ Participation:}
\noindent Participation is an essential part of this lab. Your participation grade will be based on if you come to lab prepared and on time, the number of questions you ask, initiation of any class discussions, attempts at answering any questions that I or other students pose, throughout the lab session and reflection section of each lab write-up. \\
 

\section*{Policies:}
 
\subsection*{Attendance}
 
It is important that you arrive \textbf{on time}. Your grade is made up by 10 out of 11 of these labs, you may either miss one class without an excuse or have your lowest lab grade dropped. I strongly recommend the latter. By department policy \textbf{missing more than two labs (for non-medical reason) will result in automatic failure of the course}. In cases of family emergencies, or religious holidays please notify me \textbf{before} the missed lab and I will be able to assign a make-up lab for you to complete. 
 
\section*{Accommodations}
Please speak with me if this course can be better adapted to your needs, without sacrificing the integrity of
instruction. If you have an identified disability, I encourage you to register with the Office of Disability
Services early in the semester to ensure access to any necessary resources; registration is confidential.
 
\section*{Academic honesty}
Do not falsify data. Give credit to others' work. Do not present text verbatim from other sources as if it
were your own; do not otherwise plagiarize. Please ask if you are unsure what's acceptable. 
 
\section*{Mandatory reporting}
I am required to report allegations of ``gender based misconduct, discrimination, or harassment" to
Columbia's administration. I am happy to listen and seek out resources (including confidential
counselors) on your behalf, but I cannot provide confidentiality myself.
 
\section*{Other astronomy related events at Columbia (optional):}
Public lectures and observing sessions: http://outreach.astro.columbia.edu/

\section*{Other Concerns}
Any concerns you do not wish to raise directly with me?
You may contact Prof. Laura Kay (lkay@barnard.edu), who supervises our astronomy labs.
 
\end{document}